
\linebreak

(\problem, potenciacao)
\begin{description}
%\item[a)] (1.0) Seja $w=\frac{\sqrt{3}+1+i(1-\sqrt{3})}{1-i}$, determine a forma polar de $w$ e seu argumento principal.
\item[a)] (1.5) Considerando $z=-1+i$ determine $z^{1000}$ e a ra\'iz c\'ubica \textbf{principal} de $z$.
\item[b)] (1.0) Seja $C_0$ o c\'irculo no plano complexo com centro em $z_0=i$ e raio $R=2$, mostre que 
\[\left| 2z^2-4zi-1  \right| \geq 7, \quad \forall \text{ } z\in C_0\]
\end{description}

\linebreak

(\problem, limite)
\begin{itemize}
\item[a)] (1.0) Calcule \[\lim_{z\rightarrow \infty} \frac{12z^4+1-\cos\left(|z|\right)}{3z^4+z^2+1}.\]
\item[b)] (1.5) Mostre que n\~ao existe \[\lim_{z\rightarrow 0} \frac{\operatorname{Re}(z) \cdot \operatorname{Im}(z) }{|z|^2}.\]
\end{itemize}

\linebreak

(\problem, eqCR)
Considere $f:\mathbb C \to \mathbb C$ dada por
\[f(z)= (2y(x-1)^2-y^2x) + i\cdot (y(x-1)^2), \quad z=x+i\cdot y \in \mathbb C.\]
Determine todos os pontos onde $f$ \'e deriv\'avel e calcule sua derivada nestes pontos.

\linebreak

(\problem, eqCR)
Duas fun\c c\~oes complexas $f,g:\mathbb C \to \mathbb C$ s\~ao tais que:
\begin{itemize}
\item $\operatorname{Re} (g(z)) = \operatorname{Re}(f(z))^2$,
\item $\operatorname{Im}(g(z)) = \operatorname{Im}(f(z))^2$.
\end{itemize}
Sabendo que $f$ e $g$ s\~ao deriv\'aveis em TODOS os pontos de $\mathbb C$:

\begin{itemize}
\item[a)](1.0) utilize o teorema das equa\c c\~oes de Cauchy-Riemann para mostrar que $u(x,y) = v(x,y)$ para todo $z=x+iy\in \mathbb C$.
\item[b)](1.0) mostre que $f$ deve ser uma fun\c c\~ao constante.
\end{itemize}
