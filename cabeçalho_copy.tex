\documentclass[12pt,reqno]{article}
\textwidth=14.5cm  \oddsidemargin=0.5cm
\usepackage{tikz}
\usepackage{graphicx}
\usepackage{psfrag}
\usepackage{mathrsfs}
\usepackage{color}
\usepackage{amsmath,amssymb}
\font\de=cmssi12
\font\dd=cmssi10
%%%%%%%%%%%%%%%%%%%%%%%%%%%%%%%%%%%%%%%%%%%%%%%%%%%%%%%%%%%%

\usepackage[active]{srcltx}

\numberwithin{equation}{section}

\newcommand{\twopartdef}[4]
{
	\left\{
		\begin{array}{ll}
			#1 & \mbox{,  if } #2 \\
			#3 & \mbox{, if } #4
		\end{array}
	\right.
}


\title{Introdu\c c\~ao \`a An\'alise no $\mathbb R^n$ \\ Prova $1$}
%Partially hyperbolic diffeomorphims with Null Central Lyapunov exponents and non-compact %central leaves}
\author{Prof.Gabriel Ponce \\ {\small IMECC- UNICAMP} }

\date{}      
                                    
\begin{document}

\begin{center}
{\Large Matem\'atica IV  - Fun\c c\~oes de Vari\'aveis Complexas} \\
\vspace{.2cm}
{Prof. Gabriel Ponce}
\end{center}

\noindent  \textbf{RA (Leg\'ivel)} : 
\quad 
\vspace{.5cm}

\begin{center}
\vspace{0.5cm}
\begin{tabular}{r|r|r|r|r|r|r|r|r|r}
 
1 & 2 & 3 & 4 & 5 & Total \\ % Note a separação de col. e a quebra de linhas
\hline                               % para uma linha horizontal
 \quad  \quad &\quad  \quad &\quad  \quad&\quad  \quad&\quad  \quad &\quad  \quad 
\end{tabular}
\end{center}

\noindent {\bf Observa\c c\~ao:} 
\begin{itemize}
\item[1)] Este simulado \'e formado atrav\'es de um banco de problemas selecionado pelo docente e n\~ao necessariamente tem rela\c c\~ao com formato, n\'ivel de dificuldade, ou qualquer outro aspecto das provas de MA044/Fun\c c\~oes de Vari\'avel Complexa na sua disciplina. Use este simulado apenas como uma forma adicional de estudo.
\item[2)] N\~ao se esque\c ca de verificar as hip\'oteses dos teoremas necess\'arios antes de aplica-los.
\item[3)] Justifique bem suas solu\c c\~oes.
\end{itemize}



\newpage

\begin{center}
{\Large Simulado de Matem\'atica IV  - Fun\c c\~oes de Vari\'aveis Complexas} \\
\vspace{.2cm}
{Prof. Gabriel Ponce}
\end{center}

\vspace{1cm}

 \begin{enumerate}
\item Mostre que se $|z|=2$ ent\~ao
\[\left|  \frac{1}{z^4-4z^2+3}  \right| \leq \frac{1}{3}.\]



\item Mostre que a hip\'erbole $x^2 - y^2 = 1$ pode ser escrita na forma
\[z^2+\bar{z}^2 =2.\]


\item Desenhe o conjunto dos pontos determinado pela condi\c c\~ao:
\begin{itemize}
\item[a)] $\operatorname{Re}(\overline{z}-i) = 2$;
\item[b)] $|2\overline{z} +i| = 4$.
\end{itemize}


\item Utilize as propriedades de m\'odulo j\'a demonstradas em sala para provar que, dados quaisquer n\'umeros complexos $z_1,z_2,z_3,z_4$ com $|z_3| \ne |z_4|$ tem-se:
\[\frac{\operatorname{Re}(z_1+z_2)}{|z_3+z_4|} \leq \frac{|z_1|+|z_2|}{||z_3|-|z_4||}.\]


\item  Em cada caso, encontre todas as ra\'izes em coordenadas retangulares e exiba elas como v\'ertices de um certo pol\'igono regular. Al\'em disso, identifique a raiz principal:
\begin{itemize}
\item[a)]$(-1)^{1/3}$
\item[b)] $8^{1/6}$.
\end{itemize}  



\end{enumerate}

 \end{document}